\documentclass{article}
\usepackage[utf8]{inputenc}
\usepackage{amsmath}

\title{Cálculo 2: Un Enfoque Introductorio}
\author{Tú Nombre}
\date{\today}

\begin{document}

\maketitle

\section{Introducción}
El cálculo 2 es una rama fundamental de las matemáticas que se construye sobre los conceptos presentados en el cálculo 1. En este artículo, exploraremos algunos de los temas clave del cálculo 2, incluyendo las integrales definidas, las series infinitas y las ecuaciones diferenciales.

\section{Integrales Definidas}
Una de las principales herramientas del cálculo 2 es la integral definida. La integral definida se utiliza para calcular áreas bajo curvas y resolver problemas de acumulación. Se define matemáticamente como:

\begin{equation}
\int_{a}^{b} f(x) , dx = F(b) - F(a)
\end{equation}

donde $f(x)$ es una función continua en el intervalo $[a, b]$, y $F(x)$ es una primitiva de $f(x)$.

\section{Series Infinitas}
Otro tema importante en el cálculo 2 son las series infinitas. Una serie infinita es la suma de los términos de una secuencia infinita de números. Por ejemplo, la serie geométrica es un tipo común de serie infinita dada por:

\begin{equation}
S = a + ar + ar^2 + ar^3 + \ldots = \sum_{n=0}^{\infty} ar^n
\end{equation}

donde $a$ es el primer término y $r$ es la razón común. Las series infinitas pueden converger a un valor finito o diverger hacia infinito.

\section{Ecuaciones Diferenciales}
Las ecuaciones diferenciales también desempeñan un papel importante en el cálculo 2. Una ecuación diferencial es una ecuación que involucra una función desconocida y sus derivadas. Por ejemplo, una ecuación diferencial ordinaria de primer orden se puede escribir como:

\begin{equation}
\frac{dy}{dx} = f(x, y)
\end{equation}

donde $\frac{dy}{dx}$ representa la derivada de la función desconocida $y$ con respecto a $x$, y $f(x, y)$ es una función conocida.

\section{Conclusiones}
El cálculo 2 es un área emocionante y desafiante de las matemáticas que se centra en conceptos avanzados como las integrales definidas, las series infinitas y las ecuaciones diferenciales. Estos temas tienen aplicaciones en diversas áreas, desde la física hasta la economía y la ingeniería. Esperamos que este breve artículo haya proporcionado una introducción útil al cálculo 2 y haya despertado tu interés en explorar más a fondo este fascinante campo.

\end{document}